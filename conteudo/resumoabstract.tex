%!TeX root=../tese.tex
%("dica" para o editor de texto: este arquivo é parte de um documento maior)
% para saber mais: https://tex.stackexchange.com/q/78101

% As palavras-chave são obrigatórias, em português e em inglês, e devem ser
% definidas antes do resumo/abstract. Acrescente quantas forem necessárias.

\keywords{Microservices,Extensibility,Non-functional requirements,Software Architecture}

\palavraschave{Microserviços, Extensibilidade, Requisitos não-funcionais, Arquitetura de Software}

\abstract{
    Microservices is an architecture pattern that has increase in popularity since 2010s for its benefits in scalability and context boundaries for big systems. Nowadays, tech companies use continuous delivery to create ever-growing systems that keep evolving. Microservices are usually used with it, for the ability to add new services to extend systems functionalities. However there is a lack of formal knowledge about it towards non-functional requirements besides scalability. 
    This work explores the concept of extensibility, a non-funcitonal requirement for systems that need to have its function extended in a new feature, keeping the same responsibility.
    To do so, it studies two microservices’ case studies, a Data Provider and a Data Ingestion system—each one with different architecture approaches implemented. The implementation provided runtime metrics and developer experience to compare them, mainly regarding extensibility. 
    At the end, the results of the case studies pointed to the benefits of three highlights: (i) the asynchronous approach for internal communication, (ii) the usage of metadata for integration of heterogeneous systems, and (iii) the data replication to reduce external chattiness.    
}

% O resumo é obrigatório, em português e inglês. Estes comandos também
% geram automaticamente a referência para o próprio documento, conforme
% as normas sugeridas da USP.
\resumo{
    Microserviços é um padrão arquitetural que tem aumentado em popularidade no últimos anos devido seus benefícios em relação a escalabilidade e limite de contexto para grandes sistemas. Atualmente, empresas de tecnologia usam entrega contínuoa para criar sistemas que continuam crescendo e evoluindo. Microserviços são comumentes usados junto, pela capacidade de adicionar novos serviços e extender as funcionalidades do sistema. Porém, há uma falta de conhecimento formal sobre o assunto em relação a requisitos não-funcionais além de escalabilidade.
    Esse trabalho explora o conceito de extensiblidade, um requisito não-funcional para sistemas que precisam extender suas funcionalidades, mantendo as mesmas responsabilidades.
    Para fazer isso, estudamos dois casos de estudo de microserviços, um Provedor de Dados e um sistema de Ingestão de Dados—cada um com diferentes arquiteturas implementadas. Cada implementação forneceu métricas durante a execução do sistema e de experiencia de desenvolvimento para comparar as arquiteturas, principalmente em quesito a extensibilidade.
    No final, os resultados dos caso de estudos apontaram para benefícios em três conceitos: (i) uso de comunicação assíncrona internamente entre os serviços, (ii) uso de metadados para integração de sistemas heterogêneos, e (iii) a replicação de dados para reduzir comunição desnecessária entre os serviços.
}
